%!TEX root=../hearts-revised.tex
\section{Posets with \texorpdfstring{$\Z$}{Z}-actions.}\label{posets}
\begin{modifyepigraph}{.35}
\epigraph{\japanese{為無為。事無事。味無味。}}{\japanese{老子}}
\end{modifyepigraph}

This section introduces the terminology about partially ordered groups and their actions, and then specializes the discussion to $\Z$-actions. We do not aim at a complete generality, but instead at gathering a number of useful results and nomenclature which is useful to have at hand. Among various possible choices, we mention specialized references as \cite{blyth2005lattices, glass1999partially, Fuch63} for an extended discussion of the theory of actions on ordered groups.

We begin by recalling the well known fact that the category $\Pos$ of partially ordered sets and monotone maps is cartesian closed.
Namely, the product order on the cartesian product $P\times Q$ of two posets $(P,\leq)$ and $(Q,\leq)$, given by $(x,y)\leq (x',y')$ if and only if $x\leq x'$ and $y\leq y'$ makes $(P\times Q,\leq)$ together with the projections on the factors satisfy the universal property of the product in the category of posets. Moreover, the set $\Pos(P,Q)$ of monotone maps from $(P,\leq)$ to $(Q,\leq)$ has a natural order on it given by $f\leq g$ if and only if $f(x)\leq g(x)$ for every $x$ in $P$. With this order one has 
an adjunction
\[
\Pos(P\times Q, R)\cong \Pos(P, \Pos(Q,R)).
\]
\begin{remark}
The product order is not the only standard order one puts on the product $P\times Q$ of two posets $P$ and $Q$. Another commonly used one is the lexicographic order defined by
$(x,y)\leq_{\mathrm{lex}} (x',y')$ if and only if $x< x'$ or $x=x'$ and $y\leq y'$. The lexicographic order does not make $P\times Q$ be the product of $P$ and $Q$ in the category of posets, but it still has a few peculiar properties that, as we are going to see, are relevant to the theory of slicings.
\end{remark}

\begin{remark}
\label{quotient1}
Let $(P,\leq)$ be a poset, and let $\sim$ be an equivalence relation on the set $P$. One says that $\sim$ is compatible with the order relation if $x\leq y$, $x\sim x'$ and $y\sim y'$ imply $x'\leq y'$ or $x'\sim y'$. When this happens the quotient set $P/_{\!\sim}$ inherits a order relation from $P$ by $[x]\leq [y]$ if and only if $x\leq y$ or $x\sim y$. Moreover the projection to the quotient
$P\to P/_{\!\sim}$ is a morphism of posets.
\end{remark}

\begin{definition}
A \emph{partially ordered group} (``po-group'' for short) is a pair $(G, \leq)$ consisting of a group $G$ and of a partial order relation $\leq $ on $G$ such that the group multiplication  $\cdot \colon G\times G\to G$ is a map of posets, where $G\times G$ is endowed with the product order: for any two pairs $(g,h)$ and $(g',h')$ with $g\leq g'$ and $h\leq h'$ we have $gg' \le hh'$.
\end{definition}
\begin{remark}
In the literature on the subject it is customary draw a distinction between a \emph{left} po-group and a \emph{right} po-group. We choose to ignore this subtlety, since all the po-groups we will be dealing with will be ordered by two-sided congruences.
\end{remark}
\begin{remark} If $(G,\leq)$ is a po-group, the inversion $(\firstblank)^{-1}\colon G \to G$ is an antitone antiautomorphism of groups, \ie we have that
$g\leq h$ if and only if $h^{-1}\leq g^{-1}$. Moreover
the set $G^+$ of \emph{positive} elements, i.e. the set $\{g\in G\mid 1\leq  g\}$ is closed under conjugation.
\end{remark}
\begin{example}
Let $(P,\leq)$ be a poset, and let $\mathrm{Aut}_{\Pos}(P)$ be the automorphism group of $P$ as a poset, i.e., the set of monotone bijections of $p$ into itself. Then $\mathrm{Aut}_{\Pos}(P)$ inherits an order relation by its inclusion in the poset $\Pos(P,P)$, and this makes $\mathrm{Aut}_{\Pos}(P)$ a partially ordered group. This is the standard po-group structure on $\mathrm{Aut}_{\Pos}(P)$.
\end{example}
\begin{remark}
Any group $G$ can be seen as a po-group with the trivial order relation $g\leq h$ if and only if $g=h$. It is worth noticing that on finite groups the trivial order is the only possible po-group structure. Namely, assume $g\leq h$ and let $k=g^{-1}h$. Then we have $1\leq k\leq k^2\leq k^2\leq\dots\leq k^{\mathrm{ord}(k)}=1$ and so $k=1$, i.e., $g=h$.
\end{remark}
\begin{definition}
A \emph{homomorphism} of po-groups consists of a group morphism $f\colon G\to H$ which is also a monotone mapping. This %, with the obvious choices of identities and composition, 
defines a category $\cate{PoGrp}$ of partially ordered groups and their homomorphisms.
\end{definition}
\begin{definition}\label{def:g.poset}
Let $(G,\leq)$ be a po-group. A $G$-poset is a partially ordered set $(P, \le)$ endowed with a po-group homomorphism $G\to \text{Aut}_{\Pos} (P)$ to the group of order isomorphisms of $P$ with its standard po-group structure.
\end{definition}
\begin{remark}
Equivalently, a $G$-poset is a partially ordered set $P$ together with a group action $G\times P\to P$ which is a morphism of posets, where on $G\times P$ one has the product order.
\end{remark}
\begin{example}
Every po-group $G$ is a $G$-poset with the multiplication action of $G$ on itself. 
\end{example}
\begin{remark}\label{quotient2}
An equivalence relation $\sim$ on a $G$-poset $P$ is said to be compatible with the $G$-action if $x\sim y$ implies $g\cdot x\sim g\cdot y$ for any $g$ in $G$. If $\sim$ is compatible both with the order and with the $G$-action then the quotient set $P/_{\!\sim}$ is naturally a $G$-poset with the $G$-action $g\cdot[x]=[g\cdot x]$. Moreover the projection to the quotient is a morphism of $G$-posets.
\end{remark}

We now specialize our discussion to the case $G=\Z$
\begin{definition}\label{zposet}
A $\Z$-\emph{poset} is a partially ordered set $(P,\leq)$ together with a group action 
\[
+_P\colon P\times \Z \to P \colon (x,n) \mapsto x+_Pn 
\]
 which is a morphism of partially ordered sets, when $\Z$ is regarded with its usual total order.
\end{definition}
\begin{remark}\label{trivial.but.useful}
It is immediate to see that a $\Z$-poset is equivalently the datum of a poset $(P,\leq)$ together with a monotone bijection $\rho\colon P\to P$ such that $x\leq \rho(x)$ for any $x$ in $P$. The function $\rho$ and the action are related by the identity $\rho(x)=x+_P1$.
\end{remark}
\begin{notat}
To avoid a cumbersome accumulation of indices, the action $+_P$ will be often denoted as a simple ``$+$''. This is meant to evoke in the reader the most natural example of a $\Z$-poset, given by $\mathbb{Z}$ itself, and to simplify our notation for the axioms of an action:
\[
\begin{cases}
(x +_P m) +_P n = x +_P (m+n);\\
x +_P 0 = x.
\end{cases}
\]
We will also write $x-n$ for $x+_P(-n)$.
\end{notat}
\begin{example}The poset $(\Z ,\leq)$ of integers with their usual order is a $\Z$-poset with the action given by the usual sum of integers. The poset $(\R,\leq)$ of real numbers with their usual order is a $\Z$-poset for the action given by the sum of real numbers with integers (seen as a subring of real numbers).
\end{example}
\begin{example}
Given any poset $(P,\leq)$, the poset $\Z\times_{\mathrm{lex}}P$ carries a natural $\mathbb{Z}$-action given by $(n,x)+1=(n+1,x)$, i.e., by the standard $\Z$-action on the first factor and by the trivial $\Z$-action on the second factor. 
\end{example}

\begin{remark}\label{rem.finite}
 If $(P,\leq)$ is a finite poset, then the only $\Z$-action it carries is the trivial one. Indeed, if $\rho\colon P\to P$ is the monotone bijection associated with the $\Z$-action, one sees that $\rho$ is a finite order element in $\mathrm{Aut}_{\Pos}(P)$, by the finiteness of $P$. Therefore there exists an $n\geq 1$ such that $\rho^n=\mathrm{id}_P$. It follows that, for any $x$ in $P$,
 \[
 x\leq x+1\leq\cdots\leq x+n=x
 \]
and so $x=x+1$.
\end{remark}
\begin{remark}\label{minmax}
An obvious terminology: a \emph{$G$-fixed point} for a $G$-poset $P$ is an element $p\in P$ kept fixed by all the elements of $G$ under the $G$-action. 
Clearly, an element $x$ of a $\Z$-poset $P$ is a $\Z$-fixed point if and only if $x+ 1 = x$, or equivalently $x-1=x$. From this it immediately follows that if $x\in P$ is a $\le$-maximal or $\le$-minimal element in the $\Z$-poset $P$, then it is a $\Z$-fixed point.
\end{remark}
\begin{remark}
Given a poset $P$ we can always define a partial order on the set $P_{\!\bowtie} = P\cup\{-\infty,+\infty\}$ which extends the partial order on $P$ by the rule $-\infty\leq x\leq +\infty$ for any $x\in P$. 
\end{remark}
\begin{lemma}
 If $(P,\leq)$ is a $\Z$-poset, then $(P_{\!\bowtie},\leq)$ carries a natural $\Z$-action extending the $\Z$-action on $P$, by declaring both $-\infty$ and $+\infty$ to be $\Z$-fixed points.
\end{lemma}
\begin{proof}
 Adding a fixed point always gives an extension of an action, so we only need to check that the extended action is compatible with the partial order. This is equivalent to checking that also on $P_{\!\bowtie}$ the map $x\mapsto x+1$ is a monotone bijection such that $x\leq x+1$, which is immediate. 
\end{proof}
Posets with $\Z$-actions naturally form a category $\ZPos$, whose morphisms are \emph{$\Z$-equivariant} morphisms of posets. More explicitly, if $P$ and $Q$ are $\Z$-posets with actions $+_P$ and $+_Q$, then a morphism of $\Z$-posets between them is a morphism of posets $\varphi\colon P\to Q$ such that
\[
\varphi(x+_P n)=\varphi(x)+_Q n,
\]
for any $x\in P$ and any $n\in \Z$.
\begin{remark}
If $P$ and $Q$ are $\Z$-posets, then the hom-set $\ZPos(P,Q)$ is naturally a $\Z$-poset. Namely, as we have already remarked, $\Pos(P,Q)$ is naturally a poset, and so $\ZPos(P,Q)$ inherits the poset structure as a subset. The $\Z$-action is given by $(\varphi+n)(x)=\varphi(x) +_Q n$. This makes $\ZPos$ a closed category.
\end{remark}

\begin{remark}
Every poset can be seen as a $\Z$-poset with the trivial $\Z$-action. Since every monotone mapping is $\Z$-equivariant with respect to the trivial $\Z$-action, this gives a fully faithful embedding $\Pos\to \ZPos$.
\end{remark}
\begin{lemma}\label{trivial.but.useful2}
The choice of an element $x$ in a $\Z$-poset $P$ is equivalent to the datum of a $\Z$-equivariant morphism $\varphi\colon(\Z ,\leq)\to (P,\leq)$. Moreover $x$ is a $\Z$-fixed point if and only if the corresponding morphism $\varphi$ factors $\Z$-equivariantly through $(*,\leq)$, where $*$ denotes the terminal object of $\Pos$. 
 \end{lemma}
\begin{proof}
To the element $x$ one associates the $\Z$-equivariant morphism $\varphi_x$ defined by $\varphi_x(n)=x+n$. To the $\Z$-equivariant morphism $\varphi$ one associates the element $x_\varphi=\varphi(0)$. It is immediate to check that the two constructions are mutually inverse. The proof of the second part of the statement is straightforward.
\end{proof}
\begin{lemma}
Let $\varphi\colon(\Z ,\leq)\to (P,\leq)$ be a $\Z$-equivariant morphism of $\Z$-posets. Then $\varphi$ is either injective or constant.
\end{lemma}
\begin{proof}
Assume $\varphi$ is not injective. then there exist two integers $n$ and $m$ with $n>m$ such that $\varphi(n)=\varphi(m)$. By $\Z$-equivariancy we therefore have
\[
x_\varphi+(n-m)=x_\varphi,
\]
with $n-m\geq 1$ and $x_\varphi=\varphi(0)$. The conclusion then follows by the same argument used in Remark \refbf{rem.finite}.
\end{proof}
\begin{lemma}\label{extends}
Let $\varphi\colon (P,\leq)\to (Q,\leq)$ be a morphism of $\Z$-posets. Assume $Q$ has a minimum and a maximum. Then $\varphi$ extends to a morphism of $\Z$-posets $(P_{\!\bowtie},\leq)\to (Q,\leq)$ by setting $\varphi(-\infty)=\min(Q)$ and $\varphi(+\infty)=\max(Q)$.
\end{lemma}
\begin{proof}
Since $\min(Q)$ and $\max(Q)$ are $\Z$-fixed points by Remark \refbf{minmax}, the extended $\varphi$ is a morphism of $\Z$-posets. Moreover, since $\min(Q)$ and $\max(Q)$ are the minimum and the maximum of $Q$, respectively, the extended $\varphi$ is indeed a morphism of posets, and so it is a morphism of $\Z$-posets.
\end{proof}

All of the above applies in particular to totally ordered sets. We will denote by $\Tos\subseteq \Pos$ the full subcategory of totally ordered sets, and by $\mathbb{Z}\text{-}\Tos\subseteq \ZPos$ the full subcategory of $\Z$-actions on totally ordered sets.

\begin{lemma}\label{equivalence}
Let $(P, \leq)$ be a totally ordered $\mathbb{Z}$-poset. The relation $x\sim y$ if and only if there are integers $a,b \in \mathbb{Z}$ such that $x + a \le y \le x + b $ is an equivalence relation on $P$ compatible with both the order and the $\mathbb{Z}$-action. It therefore induces a morphism of $\mathbb{Z}$-posets $P\to P/_{\!\sim}$ given by the projection to the quotient. Moreover, $P/_{\!\sim}$ is totally ordered and the $\mathbb{Z}$-action on the quotient is trivial.
\end{lemma}
\begin{proof}
Checking that $\sim$ is an equivalence relation is immediate: reflexivity is manifest; symmetry reduces to noticing that $x + a \le y \le x + b $ is equivalent to $y-b\leq x\leq y-a$; transitivity follows by  the fact that $x + a \le y \le x + b $ and $y + c \le z \le y + d$ together imply $x+ (a+ c) \le z \le x+(b + d)$. To see that $\sim$ is compatible with the order relation, let $x\leq y$ and let $x\sim x'$ and $y\sim y'$. Then there exist $a, b, c$ and $d$ in $\mathbb{Z}$ such that $x + a \le x' \le x + b $ and $y + c \le y' \le y + d$. Since $P$ is totally ordered, either $x'\leq y'$ or $y'\leq x'$. In the second case we have $y'\leq x'  \le x + b\le y+b \le y'+b-c$, and so $x'\sim y'$ by definition of the relation $\sim$. The compatibility of $\sim$ with the $\mathbb{Z}$-action is straightforward. Therefore by Remarks \refbf{quotient1} and \refbf{quotient2} we see that the projection to the quotient $P\to P/_{\!\sim}$ is a morphism of $\mathbb{Z}$-posets. Since the order on $P$ is total, so is also the order induced by $\sim$ on the quotient set. Finally, to see that the $\mathbb{Z}$ action on $P/_{\!\sim}$ is trivial, just notice that for any $x$ in $P$, we have $x\leq x+1\leq x+1$ and so $[x]+1=[x]$.
\end{proof}

\begin{remark}\label{iq=q}
If the $\mathbb{Z}$-action on the totally ordered set $P$ is trivial, then the equivalence relation from Lemma \refbf{equivalence} is trivial as well: $x\sim y$ if and only if $x=y$.
\end{remark}
\begin{lemma}\label{lemma.representatives}
Let $(P, \leq)$ be a totally ordered $\mathbb{Z}$-poset, and let $\sim$ be the equivalence relation from Lemma \refbf{equivalence}. Then either $[x]=\{x\}$ or $[x]=\mathbb{Z}\times_{\mathrm{lex}}[x,x+1)$, where $[x,x+1)=\{y\in P\mid x\leq y<x+1\}$
\end{lemma}
\begin{proof}
Let $x\in P$; then either $x=x+1$ or $x<x+1$. In the first case $x$ is a fixed point for the $\Z$-action on $P$ and so the equivalence relation $\sim$ is the trivial one: $y\sim x$ if and only if $y=x$. If $x<x+1$ then the interval $[x,x+1)$ is nonempty, as $x\in [x,x+1)$.

Let $\varphi\colon \mathbb{Z}\times_{\mathrm{lex}}[x,x+1)\to P$ the map defined by $(n,y)\mapsto y+n$. The map $\varphi$ is a morphism of $\Z$-posets. 

Indeed, if $(n,y)\leq_{\mathrm{lex}} (n',y')$ either $n<n'$ or $n=n'$ and $y\leq y'$. In the first case we have $n+1\leq n'$ and so $y+n< x+1+n\leq x+n'\leq y'+n'$, whereas in the second case we have $y+n\leq y'+n=y'+n'$. The map $\varphi$ is also injective. %Indeed, assume $\varphi(n,y)=\varphi(n',y')$. This means $y'=y+m$, where $m=n-n'$. Up to switching the role of $(y,n)$ and $(y',n')$ we may assume $m\geq 0$. Assume $m\geq 1$. From $y'=y+m$ we get $[x,x+1)\cap [x+m,x+m+1)\neq \emptyset$. In particular we have $x+m\in [x,x+1)$ and so $x\leq x+m<x+1\leq x+m$ which is impossible. So $m=0$, i.e., $n=n'$. But then $\varphi(n,y)=\varphi(n',y')$ gives also $y=y'$. As $x+n\leq \varphi(n,y)\leq x+n+1$ we see that $\varphi$ takes its values in the equivalence class of $x$. 

To conclude we only need to show that $\varphi\colon  \mathbb{Z}\times_{\mathrm{lex}}[x,x+1)\to [x]$ is surjective. Let $z\in [x]$. By definition of the equivalence relation there exist $a,b$ in $\mathbb{Z}$ such that $x+a\leq z\leq x+b$. Since $x$ is not a $\Z$-fixed point, we have $a\leq b$ and $z\in [x+a,x+b+1)$. Writing
\[
[x+a,x+b+1)=\bigcup_{n=a}^{b}[x+n,x+n+1)
\]
we see that there exists $n\in \Z$ such that $z\in [x,x+1)+n$, i.e., $z=\varphi(n,y)$ for some $y$ in $[x,x+1)$.
\end{proof}
\begin{proposition}\label{adjoint}
The fully faithful embedding $(\firstblank)^\flat\colon \Tos\to \Z\text{-}\Tos$ given by trivial $\Z$-actions on totally ordered sets has a left adjoint.
\end{proposition}
\begin{proof}
For any totally ordered $\Z$-poset $P$, let $\iota(P)$ be the $\mathbb{Z}$-poset $P/_{\!\sim}$, where $\sim$ is the equivalence relation from Lemma \refbf{equivalence}. Then $I\colon \Z\text{-}\Tos\to \Tos$ is a functor, since if $f\colon P\to Q$ is a morphism of $\mathbb{Z}$-posets then $x+a\leq y\leq x+b$ implies
\[
f(x)+a=f(x+a)\leq f(y)\leq f(x+b)=f(x)+b
\]
and so $f$ induces a well defined morphism of sets $\tilde{f}\colon \iota(P)\to \iota(Q)$. It is immediate to see that $\tilde{f}$ is actually a morphism of $\Z$-posets and that $f\rightsquigarrow \tilde{f}$ preserves identities and compositions of morphisms. Finally, to see that $I$ is a right adjoint to the trivial action embedding $\Tos\to \Z\text{-}\Tos$, let $P$ be a a totally ordered $\mathbb{Z}$-poset and $Q$ be a totally ordered set. Since $I$ is a functor, a morphism of $\mathbb{Z}$-posets $f\colon P\to Q^\flat$ induces a morphism of posets $\tilde{f}\colon \iota(P)\to \iota(Q^\flat)$. Moreover, since the $\mathbb{Z}$-action on $Q^\flat$ is trivial, we have $\iota(Q^\flat)\cong Q$, see Remark \refbf{iq=q}. Therefore, $f\mapsto \tilde{f}$ defines a map
\[
\Z\text{-}\Tos(P,Q^\flat)\to \Tos(\iota(P),Q)
\]
which we want to show is a bijection. Assume $\tilde{f}_1=\tilde{f}_2$. Then, for any $x$ in $P$ we have $f_1(x)\sim f_2(x)$ in $Q^\flat$. Since the equivalence relation on $Q^\flat$ is trivial, this means $f_1=f_2$, so $f\rightsquigarrow \tilde{f}$ is injective. Let now $\varphi$ be a morphism of posets, $\varphi\colon \iota(P)\to Q$, and let $f=\varphi\circ \pi$ where $\pi\colon P\to \iota(P)$ is the projection to the quotient. We have $\tilde{f}([x])=[f(x)]=f(x)=\varphi(\pi(x))=\varphi([x])$, and so $\varphi=\tilde{f}$, i.e., $f\rightsquigarrow \tilde{f}$ is surjective.
\end{proof}
