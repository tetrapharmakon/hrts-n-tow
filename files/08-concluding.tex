\section{Concluding remarks}
\label{concluding}
\begin{modifyepigraph}{.2}
\epigraph{That's all, folks!}{Bosko}
\end{modifyepigraph}

We have explored two classes of $J$-slicings so far: those for which $J$ has a heart, and those for which $\mathbb{Z}$ acts trivially on $J$. In this section, we show how these two cases are fundamental building blocks for all other $J$-slicings. 





{\color{green!40!black}
\begin{lemma}\label{for-the-main-theorem}
Let $\tee$ be a $J$-slicing on a stable $\infty$-category $\C$, and let $I_i\subseteq J$ be the equivalence class of $i\in J$ with respect to the equivalence relation $\sim$ of Lemma \refbf{equivalence}. For every $(\Lambda,\Upsilon)=(L\cap I_{i},U\cap I_{i})$ in $\mathcal{O}(I_i)$, let $\tee_{i;\Lambda,\Upsilon}=(\C_L\cap \C_{I_i},\C_U\cap \C_{I_i})$. Then $\tee_i\colon (\Lambda,\Upsilon)\to \tee_{i;\Lambda,\Upsilon}$ is a $I_i$-slicing of $\C_{I_i}$.
\end{lemma}
\begin{proof}
By Remark \refbf{J.induces.I}, the $J$-slicing $\tee$ induces an $\iota(J)$-semi-orthogonal decomposition of $\C$: for every equivalence class $[i]$ in $\iota(J)$ the corresponding slice in this semi-orthogonal decomposition is the subcategory $\C_{I_i}$ of $\C$ determined by the $J$-slicing $\tee$, where $I_i\subseteq J$ is the equivalence class of $i$ with respect to the equivalence reltion $\sim$, as a subset of $J$. As they are the slices of a semi-orthogonal decomposition, the subcategories $\C_{I_i}$ are stable (this can also be seen directly from the definition of the $\C_{I_i}$'s).
 As shown in Example \refbf{class.is.interval}, $I_i$ is an interval of $J$ and a sub-$\Z$-toset of $J$, simply by definition of the equivalence relation. 
For every $i$, we can therefore write $I_i=U_{i;0}\cap L_{i;1}$. By Remark \refbf{oi.vs.oj} every slicing $(\Lambda,\Upsilon)$ of $I_i$ is of the form $\Lambda=L\cap I_{i}$ and $\Upsilon=U\cap I_{i}$ for a unique slicing $(L,U)$ of $J$ with $U_{i;0}\leq U\leq U_{i;1}$. This gives an isomorphism of tosets between $\mathcal{O}(I_i)$ and the interval $[U_{i;0},U_{i;1}]$ in $\mathcal{O}(J)$.
Now, to show that $\tee_{i;\Lambda,\Upsilon}$ is a $t$-structure on $\C_{I_i}$ one verbatim repeats the proof of Lemma \refbf{to.get.slicings.on.heart} to get orthogonality of the classes and the existence of the relevant fiber sequences. Next, to show that $(\C_U\cap \C_{I_i})[1]\subseteq \C_U\cap \C_{I_i}$ notice that, since $I_i$ is an equivalence class, we have $I_i+1=I_i$ and so 
\[
(U\cap I_{i})+1=(U+1)\cap (I_i+1)\subseteq U\cap I_i.
\]
This shows that $\tee_i$ is a morphism of sets $\mathcal{O}(I_i)\to \ts(\C_{I_i})$ and it is immediate to see that this is actually a morphism of tosets. Finally, $\Z$ equivariance is obtained by noticing that $\Upsilon+1=(U\cap I_i)+1=(U+1)\cap I_i$ and $\Lambda+1=(L+1)\cap I_i$, so that
\[
\tee_{i;\Lambda+1,\Upsilon+1}=(\C_{L+1}\cap \C_{I_i},\C_{U+1}\cap \C_{I_i})=(\C_L[1]\cap \C_{I_i},\C_U[1]\cap \C_{I_i})=\tee_{i;\Lambda,\Upsilon}[1],
\]
where we used that $\C_{I_i}$ is a stable subcategory of $\C$ and so $\C_{I_i}=\C_{I_i}[1]$.
\end{proof}

We can now state and prove our main result, summarising and putting together the various pieces constructed so far. To make a self-standing statement, we explicitly recall the definition of the equivalence relation $\sim$ from Lemma \refbf{equivalence} in the statement of the theorem below.
\begin{theorem}\label{conclusion}
Let $(J,\leq)$ a $\Z$-toset. A finite type $J$-slicing on a stable $\infty$-category $\C$ is equivalent to the following data:
\begin{enumerate}
\item[(i)] a finite type semi-orthogonal decomposition of $\C$ whose slices $\C_{[x]}$ are stable  $\infty$-subcategories of $\C$ indexed by equivalence classes in $J$ with respect to the equivalence relation $x\sim y$ if and only if there exist integers $n_1$ and $n_2$ with $x+n_1\leq y\leq x+n_2$;
\item[(ii)] a bounded $t$-structure $\tee_{[x]}$ on $\C_{[x]}$ for each  $[x]$ in $J/_{\!\sim}$;
\item[(iii)] a finite type abelian $[x,x+1)$-slicing on $\C_{[x]}^\heart$ for every $[x]$ in $J/_{\!\sim}$ such that $x$ is not a fixed point of the $\Z$-action on $J$.
\end{enumerate}
\end{theorem}
\begin{proof}
By Lemma \refbf{for-the-main-theorem}, a $J$-slicing on a stable $\infty$-category induces semi-orthogonal decomposition of $\C$ whose slices $\C_{I_i}$ are stable  $\infty$-subcategories of $\C$ indexed by equivalence classes $I_i$ in $J$ with respect to the equivalence relation $\sim$, together with $I_i$-slicings of these subcategories.
If $i$ is a fixed point for the $\Z$-action on $J$, then $I_i=\{i\}$ and $\mathcal{O}(I_i)=\ordered{1}$ so that a $I_j$-slicing is trivial. On the other hand, by Example \refbf{I.has.a.heart}, precisely when $i$ is not a fixed point of the $\Z$-action the interval $I_i$ has a heart $I_i^\heart$ which can be identified with the interval $[i,i+1)$ of $J$. Therefore, by \aprop\refbf{J-to-t} an $I_i$-slicing on $\C_{I_i}$ induces a $t$-structure on $\C_{I_i}$ together with an abelian $[i,i+1)$-slicing on the standard heart $\C_{I_i}^\heart$. Moreover, by \aprop\refbf{converse.if.finite}, in the finite type case an $I_i$-slicing on $\C_{I_i}$ is precisely equivalent to this datum of a bounded $t$-structure on $\C_{I_i}$ with an abelian $I_i^\heart$-slicing on $\C_{I_i}^\heart$. 
\par
Vice versa, given the data (i)-(iii), we want to define a $J$-slicing of $\C$. For every upper set $U$ of $J$, we can write 
\[
U=\bigcup_{[i]\in J/\sim}U\cap I_i
\]
and, by Prop. \refbf{converse.if.finite} again, the data (i)-(iii) define $\infty$-subcategories $\C_{U\cap I_i}\subseteq \C_{[i]}$ of $\C$. Defining $\C_U$ as the extension closed $\infty$-subcategory of $\C$ generated by the $\C_{U\cap I_i}$'s provides a map $\tee\colon \O(J)\to \ts(\C)$ which is immediate to see it is montone and $\mathbb{Z}$-equivariant and so is a $J$-slicing of $\C$. Moreover, this $J$-slicing is of finite type as all the slicings provided by the data are of finite type, and manifestly this way of constructing  finite type $J$-slicings out of
data (i)-(iii) provides an inverse to the construction described in the first part of the proof.  
\end{proof}
}








