\documentclass{article}

\usepackage{todonotes}

\begin{document}
% \newpage 
 \begin{itemize}
  \item By far the biggest flaw in the exposition is the omission of the definition of a “pullout” square.
  \item Again, as a matter of the reader’s convenience, it would probably be
  good to isolate the statement of the Sator lemma in a reference-able
  proposition type format.
  \item the authors are not very clear on which aspects of
  the theory actually depend on the stable infty-category itself as opposed
  to only its triangulated homotopy category. 
 \end{itemize}
 These issues have all been addressed in an additional subsection, §1.1.
 \begin{itemize}
  \item Definition 5.18 appears to be completely unnecessary to me;
 \end{itemize}
 The definition has been removed (red text at page 30). Instead, we opted for a lighter explanation of what we mean by abelian category in $\infty$-categorical world.
 \begin{itemize}
  \item I do not understand the latter half of the proof of lemma 7.8, beginning
  with “let X be an object of C and consider...”
 \end{itemize}
 This has been addressed adding a couple of lines just before ``let $X$ be an object'' (green text at page 42)
 \begin{itemize}
  \item The statement and “proof” of theorem 8.2 is clumsy.
  \item In fact I think the authors do themselves a slight disservice by promoting their paper as part of the theory of stable infty-categories, when really
  much of what they do is only at the level of triangulated categories,
  and thus might attact a wider readership (there is basically nowhere in
  the paper where actual higher categorical arguments are employed).
 \end{itemize}
 8.2 now has a proof. §1.1 frees the reader from actually using $\infty$-categories, but explains how the main result (the `Rosetta stone') holds only for such categories.

\end{document}