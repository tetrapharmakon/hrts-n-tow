\documentclass{article}

\usepackage{todonotes}

\begin{document}
% \newpage 
\noindent Dear Editor,
\vskip .5 cm

\noindent All of the Referee's comments have been implemented in the revised version of the manuscript. For the Referee's convenience, the modified parts of the manuscript are highlighted in green in the revised version.

More in detail:

\vskip .5 cm

 \begin{itemize}
  \item By far the biggest flaw in the exposition is the omission of the definition of a “pullout” square.
  \item Again, as a matter of the reader’s convenience, it would probably be
  good to isolate the statement of the Sator lemma in a reference-able
  proposition type format.
  \item the authors are not very clear on which aspects of
  the theory actually depend on the stable infty-category itself as opposed
  to only its triangulated homotopy category. 
 \end{itemize}
 These three issues have all been addressed in the additional Subsection, \S1.1.
 \begin{itemize}
  \item Definition 5.18 appears to be completely unnecessary to me [\dots] it would probably be a lot less confusing (and then also save the need-
less remark 5.19) to just say that an abelian $\infty$-category is an abelian
category regarded as an $\infty$-category.;
 \end{itemize}
 We closely followed the Referee's suggestion here. Additionally, what was our Definition 5.18 has now been changed into a Remark.
 \begin{itemize}
  \item I do not understand the latter half of the proof of lemma 7.8, beginning
  with “let $X$ be an object of $\mathcal{C}$ and consider...”
 \end{itemize}
A few lines have been added in the proof to make the argument hopefully clearer.
 \begin{itemize}
  \item The statement and “proof” of theorem 8.2 is clumsy.
 \end{itemize}
 The whole Section 8 has now been reworked in a less colloquial statement/proof format.
  
  \begin{itemize}
  \item In fact I think the authors do themselves a slight disservice by promoting their paper as part of the theory of stable $\infty$-categories, when really
  much of what they do is only at the level of triangulated categories,
  and thus might attract a wider readership (there is basically nowhere in
  the paper where actual higher categorical arguments are employed).
 \end{itemize}
In the additional Subsection, \S1.1. frees the reader from actually using $\infty$-categories, clearly stating that all the statements presented in the manuscript admit an immediate translation in the more widely familiar setting of triangulated categories; at the same time we present motivation for our choice of writing the proofs in the language of stable $\infty$-categories.


\end{document}